\documentclass{article}

\author{JOHN BWALE}
\title{COMPUTER GRAPHICS}
\date{TODAY}

\begin{document}GENERAL MIKTEX INFORMATION
ReportDate: 2023-11-01 22:06:08
CurrentVersion: 23.10
SetupDate: 2023-10-19 23:06:00
SetupVersion: 23.10
Configuration: Regular
GitInfo: 3f4a6a7 / 2023-10-04 18:18:19
OS: Windows 10.0.16299
SharedSetup: no
LinkTargetDirectory: C:\Users\JOHN\AppData\Local\Programs\MiKTeX\miktex\bin\x64
PathOkay: yes
LastUpdateCheck: 2023-10-31 12:42:47
LastUpdate: not yet
LastUpdateDb: 2023-10-31 12:38:53
SystemAdmin: yes
RootPrivileges: no
AdminMode: no
Root0: C:\Users\JOHN\AppData\Roaming\MiKTeX
Root1: C:\Users\JOHN\AppData\Local\MiKTeX
Root2: C:\Users\JOHN\AppData\Local\Programs\MiKTeX
UserInstall: C:\Users\JOHN\AppData\Local\Programs\MiKTeX
UserConfig: C:\Users\JOHN\AppData\Roaming\MiKTeX
UserData: C:\Users\JOHN\AppData\Local\MiKTeX
Invokers: .../explorer/Code/Code

ERROR DETAILS
Error: invalid stoi argument
GENERAL MIKTEX INFORMATION
ReportDate: 2023-11-01 22:06:08
CurrentVersion: 23.10
SetupDate: 2023-10-19 23:06:00
SetupVersion: 23.10
Configuration: Regular
GitInfo: 3f4a6a7 / 2023-10-04 18:18:19
OS: Windows 10.0.16299
SharedSetup: no
LinkTargetDirectory: C:\Users\JOHN\AppData\Local\Programs\MiKTeX\miktex\bin\x64
PathOkay: yes
LastUpdateCheck: 2023-10-31 12:42:47
LastUpdate: not yet
LastUpdateDb: 2023-10-31 12:38:53
SystemAdmin: yes
RootPrivileges: no
AdminMode: no
Root0: C:\Users\JOHN\AppData\Roaming\MiKTeX
Root1: C:\Users\JOHN\AppData\Local\MiKTeX
Root2: C:\Users\JOHN\AppData\Local\Programs\MiKTeX
UserInstall: C:\Users\JOHN\AppData\Local\Programs\MiKTeX
UserConfig: C:\Users\JOHN\AppData\Roaming\MiKTeX
UserData: C:\Users\JOHN\AppData\Local\MiKTeX
Invokers: .../explorer/Code/Code

ERROR DETAILS
Error: invalid stoi argument
GENERAL MIKTEX INFORMATION
ReportDate: 2023-11-01 22:06:08
CurrentVersion: 23.10
SetupDate: 2023-10-19 23:06:00
SetupVersion: 23.10
Configuration: Regular
GitInfo: 3f4a6a7 / 2023-10-04 18:18:19
OS: Windows 10.0.16299
SharedSetup: no
LinkTargetDirectory: C:\Users\JOHN\AppData\Local\Programs\MiKTeX\miktex\bin\x64
PathOkay: yes
LastUpdateCheck: 2023-10-31 12:42:47
LastUpdate: not yet
LastUpdateDb: 2023-10-31 12:38:53
SystemAdmin: yes
RootPrivileges: no
AdminMode: no
Root0: C:\Users\JOHN\AppData\Roaming\MiKTeX
Root1: C:\Users\JOHN\AppData\Local\MiKTeX
Root2: C:\Users\JOHN\AppData\Local\Programs\MiKTeX
UserInstall: C:\Users\JOHN\AppData\Local\Programs\MiKTeX
UserConfig: C:\Users\JOHN\AppData\Roaming\MiKTeX
UserData: C:\Users\JOHN\AppData\Local\MiKTeX
Invokers: .../explorer/Code/Code

ERROR DETAILS
Error: invalid stoi argument


\section{SECTION (A) CHOICE MAKING }
\subsection{Which of the following is NOT a primary colour in the RGB colour model?}
\begin{itemize}
\item A.RED
\item B.GREEN
\item C.BLUE
\item D.YELLOW
\item ans (D) Yellow is NOT a primary color in the RGB color model. In the
RGB model, the primary colors are Red, Green, and Blue.
\subsection{ When light travels from air into water, it generally:}
\item A. Slows down and bends away from the normal.
\item B. Speeds up and bends away from the normal.
\item C. Slows down and bends toward the normal.
\item D. Speeds up and bends toward the normal
\item  ans (B) When light travels from air into water, it generally slows down
and bends toward the normal. This phenomenon is known as refraction.
\subsection{In ray tracing, what is the term used to describe
the process of determining which objects in the scene
are visible from a specific point of view?}
\item A.Shading
\item B.Reflection
\item C.Shadowing
\item D.Ray intersection
\item ans (C) In ray tracing, the term used to describe the process of determining
which objects in the scene are visible from a specific point of view is
”Ray intersection.” Ray tracing involves casting rays from the camera’s
viewpoint and determining which rays intersect with objects in the scene.
\subsection{Which rendering technique is primarily used to
simulate the way light interacts with translucent materials such as glass or water?}
\item A. Ray casting
\item B.Photon mapping
\item C.Refraction
\item D.Specular reflection
\item ans (D) The rendering technique primarily used to simulate the way light
interacts with translucent materials such as glass or water is ”Refraction.”
Refraction is the bending of light as it passes through different materials
with varying refractive indices, and it is crucial for simulating realistic
effects in transparent or translucent materials.
\section{SECTION (B) RAY TRACING AND ANIMATION}
\subsection{Briefly explain the concept of specular reflection
and provide an example of a real-world situation where
it occurs.}
\item Specular reflection is the phenomenon in which light reflects off a surface
in a concentrated and mirror-like manner, resulting in a highlight. This
type of reflection occurs when light rays hit a smooth and polished surface,
such as glass, water, or a shiny metal, and bounce off at consistent angles.
An example of specular reflection can be seen when sunlight reflects off the
surface of a calm lake, creating a bright and well-defined glint or highlight
on the water’s surface.
\subsection{ Define ”ray tracing” and ”ray casting” in the
context of computer graphics. Explain how they differ.}
\item Ray tracing is a rendering technique that simulates the path of individual
rays of light as they interact with objects in a scene. It takes into acccount various optical phenomena, such as reflection, refraction, shadows,
and complex lighting effects. Ray tracing can produce highly realistic
and detailed images, but it is computationally intensive, as it traces rays
through the scene to calculate the color of each pixel. It is commonly used
for generating photorealistic images and animations.Ray casting, on the other hand, is a simplified rendering technique that
traces rays from the camera’s viewpoint into the scene to determine which
objects are visible from that viewpoint. It is less computationally intensive than full ray tracing and is often used for basic scenes and real-time
applications like video games. Ray casting doesn’t account for complex
lighting effects, such as reflections and refractions, and is primarily used
for visibility calculations.
\subsection{Describe the term ”interpolation” as it relates
to animation and its importance in creating smooth
motion.}
\item  Interpolation in animation refers to the process of generating intermediate
frames between keyframes to create smooth motion. Keyframes are specific frames in an animation sequence that define key positions, rotations,
and other attributes of objects or characters. Interpolation is essential
for creating fluid and realistic animations. It calculates the in-between
frames to smoothly transition between keyframes, ensuring that the motion appears natural and continuous. Without interpolation, animations
would appear jerky and less lifelike. Different types of interpolation methods, such as linear, cubic, or bezier interpolation, can be used to control
the ease-in, ease-out, and overall timing of animation sequences, allowing
animators to achieve the desired level of realism and artistic expression.
\section{ SECTION (C) PROBLEM SOLVING RAY
OF LIGHT}
\item  Consider a ray tracing scenario where a ray of
light travels from air (n1 = 1.00) into a glass block (n2
= 1.50). The incident angle is 30➒ . Calculate the angle
of refraction using Snell’s Law. Show all your work.
\end{itemize}
\end{document}
